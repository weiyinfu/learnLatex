\documentclass{article}
\begin{document}
输入完美的Tex公式
##间距。
间距为quad,一个quad包含6个thinspace,`\,`表示+1个thinspace,`\!`表示-1个thinspace。
例如,`\sum_{p\rm\;prime}f(p) = \int_{t>1}f(t)d\pi(t).`,$f(t)$和$d\pi(t)$之间的距离应该大一点。
$$\sum_{p\rm\;prime}f(p) = \int_{t>1}f(t)d\pi(t).$$
例如,`\int\!\!\!\int_D dx\,dy`,两个积分号之间距离应该小一点。
$$\int\!\!\!\int_D dx\,dy$$

##使用分解符
`\left|-x\right|=\left|+x\right|`,单纯写成|-x会使程序以为|和x都是操作数,`-`是操作符,造成`-`和前后两个操作数之间距离较大。
$\left|-x\right|=\left|+x\right|$

##不同的省略号
通常应该把 `\cdots` 用在 +,-,= 这类“高脚符号”之间,而把 `\ldots` 用在逗号这样的“矮子” 符号之间。不幸的是我发现很多书籍错用了这两种符号,或者是因为 它的作者使用的程序无法区分这两种符号。
`x_1+x_2+\cdots+x_n$\quad and \quad $x_1,\ldots, x_n`
$x_1+x_2+\cdots+x_n$\quad and \quad $x_1,\ldots, x_n$
`x_1+x_2+\ldots+x_n$\quad and \quad $x_1,\cdots, x_n`
$x_1+x_2+\ldots+x_n$\quad and \quad $x_1,\cdots, x_n$

\end{document}
